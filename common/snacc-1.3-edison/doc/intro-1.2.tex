% file: .../doc/intro-1.2.tex

% $Header: /data/cvs/prototype/story/src/public/tools/snacc-1.3/doc/intro-1.2.tex,v 1.1 2006/10/16 09:07:55 joywu Exp $
% $Log: intro-1.2.tex,v $
% Revision 1.1  2006/10/16 09:07:55  joywu
% no message
%
% Revision 1.2  1997/02/16 16:49:30  rj
% made return *this after calling abort()'' a compile time option.
%
% Revision 1.1  1997/01/01 22:47:30  rj
% first check-in
%

\chapter{\label{intro-1.2}Introduction}

Snacc compiles ASN.1 \cite{X.208} (Abstract Syntax Notation One) modules into C, C++, CORBA IDL \cite{corba} or type tables.
The generated C or C++ code contains equivalent data structures and
routines to convert values between the internal (C or C++)
representation and the corresponding BER \cite{X.209} (Basic Encoding Rules) format.
The name ``snacc'' is an acronym for ``Sample Neufeld ASN.1 to C/C++ Compiler''.

This is release 1.2rj\footnote{\emph{1.2} since it is the successor of 1.1 and \emph{rj} as i don't think that I'm the only one who worked on Snacc.} of Snacc.
This chapter lists only the differences to the original author's last release, Snacc~1.1.
The list in this chapter is incomplete---for a more thorough enumeration, see the file {\ufn .../ChangeLog}.

New features include:
\begin{itemize}
\item
  The output files generated get names derived from their input file's name, with only the suffix replaced.
  This eases makefile writing, as now you can use simple suffix rules or other forms of filename pattern matching.
  The old behaviour, where the output files got their name from the ASN.1 module name, can be retained by using the {\ufn -mm} command line switch to {\ufn snacc}.
\item
  The C++ backend generates code with a much more complete set of constructors, destructors and assignment operators.
\item
  The C++ backend can supply the generated C++ classes with \emph{meta} information about their own structure.
  This information can be used to build interpreted interfaces; the Snacc 1.2rj distribution contains a Tcl interface that uses this meta information as well as a Tcl script (that uses the Tcl interface) for a simple editor.
\item
  Snacc has successfully been ported to Linux and Alpha OSF/1, and should be both byte order and 64 bit clean.
\item
  A new backend that generates CORBA IDL (work-in-progress, not even alpha)
\end{itemize}

The makefiles have been rewritten.
The old ones removed the {\ufn .o} files after successful compilation, and thus, for every tiny code change, a full recompilation took place!
With the new set of makefiles, only those files that need to be remade are.
Following usual conventions, the phony targets depend, check, install, clean and clobber have been added.
\newline
NOTE: the makefiles still are not safe for parallel making.
\\
See Appendix~\ref{makefile-sect} for some explanations of some the makefile tricks.
\\
If you've got problems with the makefiles, please refer to the appendix!

The 1.1 release used five config.h files, and three almost identical copies of the ASN.1 C library.
They have all been merged in the file {\ufn \dots/snacc.h}.
A very small number (currently three) of compilation switches has been put into {\ufn \dots/policy.h}.

The previous release produced huge virtual inline functions.
Due to their size, these inlines wouldn't get inlined anyway.
Virtual functions don't get inlined (they get referenced via pointer in the virtual function table).
Due to their size they wouldn't offer any speed advantage (the function call overhead diminishes).
Instead, the compiler generated static functions in every {\ufn .C} file were the {\ufn .h} file is included!
This inflates the {\ufn .o} files and executables real quick (I'm speaking of MBytes per executable).
These functions have been turned into normal functions.

\section{\label{install-section}Configuring and Installing Snacc}

First of all, if you haven't already done so, un-archive Snacc to
produce the directory {\ufn snacc-1.2rj.\textit{patchlevel}} and its contents.
The directory {\ufn snacc-1.2rj.\textit{patchlevel}} will henceforth be referred to as ``{\ufn \dots}''.
The following tools are required to compile Snacc:
\begin{itemize}
  \item {\ufn make} (GNU {\ufn make} is recommended)
  \item {\ufn patch} (for a tiny patch in {\ufn \dots/c-lib/})
  \item {\ufn makedepend} or a look-alike (all of them have their advantages and disadvantages, it is hard to recommend any of them, see below)
  \item {\ufn lex} or GNU's {\ufn flex} ({\ufn flex} is recommended)
  \item {\ufn yacc} or GNU's {\ufn bison} ({\ufn bison} is recommended)
  \item a C compiler (it doesn't have to support ANSI, K\&R will do)
\end{itemize}

Some versions of {\ufn yacc} may choke due to the large size of the
{\ufn parse-asn1.y} file, however, we have had no problems with {\ufn bison}.
Our {\ufn yacc} grammar for ASN.1 has 61 shift/reduce errors and 2
reduce/reduce errors.  Most of these errors were introduced when
certain macros were added to the compiler.  Some of the shift/reduce
errors will require you to follow the offending macro in the ASN.1
module with a semi-colon.  The reduce/reduce errors were introduced by
macros that have ``Type or Value Lists'' because the NULL Type and
NULL values use the same symbol, ``NULL''.  This is not a problem
since no real processing is done with the macros in question at the
present.

{\ufn Lex} will work for the {\ufn lex-asn1.l} file but {\ufn flex} will typically
produce a smaller executable.  Most versions of {\ufn lex} have a small
maximum token size that will cause problems for long tokens in the
ASN.1 source files, such as quoted strings.  To avoid this problem,
increase the {\C YYLMAX} value in the generated {\ufn lex-asn1.c} file to at least
2048.  {\ufn Flex} does not seem to have this problem.

The configuration process has been simplified (at least for the installer of Snacc ;-) by the use of GNU autoconf.

The only file has may have to be edited is {\ufn \dots/policy.h}.
It contains a few compilation switches you may want to toggle.

The behaviour of makedepend has been changed from X11R5 to R6.
The new version keeps the source files' dirname and replaces the suffix only, the old version removed the dirname.
The makefiles expect the new behaviour.
If you've still got the R5 makedepend, the compiler's dependencies will be useless.
(If you only install the code and don't make any source code changes, this won't hurt you.)
If you haven't got {\ufn makedepend}, you can use any of the look-alikes, which often are {\ufn sh}-scripts calling the compiler with the {\ufn -M}-switch.
If you don't plan to make any source code changes, you can replace {\ufn makedepend} with {\ufn /bin/true}.
\newline
Warning: MIT X11's makedepend is broken, in both R5 and R6.
It silently does not produce any output for many of Snacc's C++ files (in {\ufn \dots/c++-lib/}).
\newline
The C compiler called with the {\ufn -M}-switch gives much better results, but is \emph{much} slower.

The Snacc compiler and library C code has been written to support ANSI or non-ANSI C\@.
The configuration script tries to find out whether your C compiler understands ANSI C\@.

The configuration script generates two files:
\begin{description}
  \item[{\ufn \dots/makehead}] gets included by all makefiles.
    It contains a lot of definitions used by make.
  \item[{\ufn \dots/config.h}] contains all the machine, operating system, compiler and environment dependent settings.
    It is included by {\ufn \dots/snacc.h}.
\end{description}

The C++ runtime library is known to compile with both {\ufn gcc 2.5.8} and {\ufn gcc 2.6.3}.
The latter has the {\C bool} type built-in (which the configuration script automatically detects).

Finally, to compile {\ufn snacc} and the C and C++ runtime libraries,
type the following at the shell prompt:

\begin{verbatim}
%1 cd snacc-1.2rj.*
%2 ./configure
%3 make
\end{verbatim}

If you wish to install only the C (including type tables) or only the
C++ versions of the library, type {\ufn make c} or {\ufn make c++},
respectively, instead of {\ufn make}.  If the make succeeds, the
{\ufn snacc} binary should be present as {\ufn \dots/compiler/snacc},
the C runtime libraries, {\ufn libasn1csbuf.a},
{\ufn libasn1cebuf.a}, {\ufn libasn1cmbuf.a} and {\ufn libasn1ctbl.a}, should be in
{\ufn \dots/c-lib/} and the C++ runtime library, {\ufn libasn1c++.a}
(and, if you compiled with the Tcl option enabled,
another runtime library, {\ufn libasn1tcl.a}),
should be in {\ufn \dots/c++-lib/}.
The type table tools,
{\ufn ptbl}, {\ufn pval} and {\ufn mkchdr}, will be in their respective directories under {\ufn \dots/tbl-tools/}.

After compiling the libraries, you can test the library routines by calling {\ufn make check}
(or by calling {\ufn make c-check} or {\ufn make c++-check} to test the C or C++ library routines only, respectively).

Manual pages that contain information on running {\ufn snacc} and the type table tools can be found in {\ufn \dots/doc/}.  

To install Snacc, you can call {\ufn make install} (or {\ufn make c-install} or {\ufn make c++-install}, respectively).
This installs the snacc compiler binary, the libraries, the {\ufn .h} and {\ufn .asn1} files, the type table tools, as well as the manual pages into the usual directories.

To remove the {\ufn .o} and other intermediate files, you can call {\ufn make clean}.
To remove the binaries, libraries and all other generated files as well, call {\ufn make clobber}.

\section{\label{run-section}Running Snacc}

Snacc is typically invoked from the shell command line and has the synopsis:
\begin{verbatim}
snacc [-h] [-P] [-t] [-e] [-d] [-p] [-f]
      [ -c | -C | -idl | -T <table output file>]
      [-meta <module>.<type>[,...]] [-mA | -mC]
      [-tcl <module>.<type>[,...]]
      [-u <useful types ASN.1 file>]
      [-mm] [-mf <max file name len>]
      [-l <neg number>]
      [-novolat]
      <ASN.1 file list>
\end{verbatim}

Snacc generates C or C++ source code for BER encode and decode
routines as well as print and free routines for each type in the given
ASN.1 modules.  Alternatively, snacc can produce type tables that can
be used for table based/interpreted encoding and decoding.  The type
table based methods tend to be slower than their C or C++ counterparts
but they tend use less memory (table size vs. C/C++ object code).

Snacc may also be used to generate CORBA IDL\@.
This part of Snacc is very new and I would rate it as pre-alpha.

The {\ufn -meta}, {\ufn -mA}, {\ufn -mC} and {\ufn -tcl} options are only present when the Tcl and Tk libraries where found at configuration time.

Most of the 1990 ASN.1 features are parsed although some do not affect
the generated code.  Fairly rigourous error checking is performed on
the ASN.1 source; any errors detected will be reported (printed to
{\C stderr}).

Each file in the ASN.1 file list should contain a complete ASN.1
module.  ASN.1 modules that use the IMPORTS feature must be compiled
together (specify all necessary modules in the ASN.1 file list).  The
generated source files will include each module's header file in the
command line order.  This makes it important to order the modules from
least dependent to most dependent on the command line to avoid type
ordering problems. Currently, snacc assumes that each ASN.1 file
given on the command line depends on all of the others on the command
line.  No attempt is made to only include the header files from
modules referenced in the import list for that module.

If the target language is C, snacc will generate a {\ufn .h} and {\ufn .c} file for each specified ASN.1 module.
If the target language is C++, snacc will generate a {\ufn .h} and {\ufn .C} file for each module.
If the target language is CORBA IDL, snacc will generate an {\ufn .idl} file for each module.
The generated file names will be derived from the module's filenames, or from the
module names if the {\ufn -mm} command line switch has been given.

The command line options are:

\begin{description}
  \item[--h    ] {short for ``help'', prints a synopsis of snacc and exits.}

  \item[--c    ] {causes snacc to generate C source code.
  This is the default behaviour of snacc if neither of the {\ufn -c} or {\ufn -C} options are given.
  Only one of the {\ufn -c}, {\ufn -C}, {\ufn -idl} or {\ufn -T} options should be specified.}

  \item[--C    ] {causes snacc to generate C++ source code.}

  \item[--novolat] {causes snacc to generate C++ ``{\C return *this}''
  after calling {\C abort()}. (Some broken compilers don't know about
  volatile functions, or their abort() isn't correctly typed.)}

  \item[--idl  ] {causes snacc to generate CORBA IDL source code.}

  \item[--T \emph{file}] {causes snacc to generate type tables and write them to the given file \emph{file}.}

  \item[--meta \emph{types}]
    causes snacc to generate C++ classes with type meta information.
    Requires C++ functionality and therefore implies {\ufn -C} (C++ code generation).

  The \emph{types} denote the PDUs and have the following syntax: a comma-separated list of pairs of: module name, a dot, and a type name from that module. (Example: {\ufn snacc -tcl M1.T-a,M-2.Tb mod1.asn1 m2.asn1})

  \item[--mA \textnormal{and} --mC]
    causes the metacode to use identifiers as defined in the ASN.1 source files or as used in the generated C++ code, respectively.
    (It defaults to {\ufn -mC}.)

  \item[--tcl \emph{types}]
    causes snacc to generate functions for a Tcl interface.
    Needs the type meta information and thus implies {\ufn -meta} (see above).
    The {\ufn -meta} option can and should be omitted, the \emph{types} are as for the {\ufn -meta} option (the \emph{types} arguments are additive, in case you specify both options).

  \item[--P    ] {causes snacc to print the parsed ASN.1 modules to {\C stdout} after the types have been linked, sorted, and processed.
  This option is useful for debugging snacc and observing the modifications snacc performs on the types to make code generation simpler.}
\end{description}

The options, {\ufn -t, -v, -e, -d, -p,} and {\ufn -f} affect
what types and routines go into the generated source code.
These options do not affect type table generation. If none of
them are given on the command line, snacc assumes that all of them are
in effect.  For example, if you do not need the Free or Print
routines, you should give the {\ufn -t -v -e -d} options to snacc.
This lets you trim the size of the generated code by removing
unnecessary routines; the code generated from large ASN.1
specifications can produce very large binaries.

\begin{description}
\item[--t    ]
  causes snacc to generate type definitions in the target language for each ASN.1 type.

\item[--v    ]
  causes snacc to generate value definitions in the target language for each ASN.1 value.
  Currently value definitions are limited to INTEGERs, BOOLEANs and OBJECT IDENTIFIERs.

\item[--e    ]
  causes snacc to generate encode routines in the target language for each ASN.1 type.

\item[--d    ]
  causes snacc to generate decode routines in the target language for each ASN.1 type.

\item[--p    ]
  causes snacc to generate print routines in the target language for each ASN.1 type.

\item[--f    ]
  causes snacc to generate free routines in the target language for each ASN.1 type.
  This option only works when the target language is C\@.
  The free routines hierarchically free C values.
  A more efficient approach is to use the provided nibble-memory system.
  The nibble memory permits freeing an entire decoded value without traversing the decoded value.
  This is the default memory allocator used by snacc generated decoders.
  See file {\ufn \dots/c-lib/inc/asn-config.h} to change the default memory system.
  For more information on the memory management see Section~\ref{lib-mem-C-section}.

\item[--u \emph{file}]
  causes snacc to read the useful types definitions from the ASN.1 module in file \emph{file} for linking purposes.
  For some ASN.1 specifications, such as SNMP, the useful types are not needed.
  The types in the given useful types file are globally available to all modules; a useful type definition is overridden by a local or explicitly imported type with the same name.
  The useful type module can be found in {\ufn \dots/asn1specs/asn-useful.asn1} and contains:
  \begin{itemize}
    \setlength{\itemsep}{0pt}
    \setlength{\parsep}{0pt}
    \nspace{0}
    \item ObjectDesccriptor
    \item NumericString
    \item PrintableString
    \item TeletexString
    \item T61String
    \item VideoTexString
    \item IA5String
    \item GraphicString
    \item ISO646String
    \item GeneralString
    \item UTCTime
    \item GeneralizedTime
    \item EXTERNAL
  \end{itemize}

\item[--mm]
  This switch is supplied for backwards compatibility.
  Snacc versions 1.0 and 1.1 produced files with names generated from the ASN.1 module name contained in the input file.
  Snacc 1.2rj by default retains the input file name and replaces the suffix only.
  The new behaviour makes {\ufn makefile} writing easier, as with modern {\ufn make}s, pattern matching can be used.

\item[--mf \emph{number}]
  causes the names of the generated source files to have a maximum length of \emph{number} characters, including their suffix.
  The \emph{number} argument must be at least 3.
  This option is useful for supporting operating systems that only support short file names.
  A better solution is to shorten the module name of each ASN.1 module.

\item[--l \emph{number}]
  this is fairly obscure but may be useful.
  Each error that the decoders can report is given an id number.
  The number \emph{number} is where the error ids start decreasing from as they are assigned to errors.
  The default is -100 if this option is not given.
  Avoid using a number in the range -100 to 0 since they may conflict with the library routines' error ids.
  If you are re-compiling the useful types for the library use -50.
  Another use of this option is to integrate newly generated code with older code; if done correctly, the error ids will not conflict.

\end{description}

Since ASN.1 has different scoping rules than C and C++, some name munging
is done for types, named-numbers etc. to eliminate conflicts.
Some capitalization schemes were chosen to fit common C programming
style.  For all names, dashes in the ASN.1 source are converted to
underscores.  See Sections \ref{naming-C-section} and \ref{naming-C++-section}
for more naming information.

If the {\ufn -mm} switch has been given, the module name is used as a base name for the generated source file
names.  It will be put into lowercase and dashes will be replaced with
underscores.  Module names that result in file names longer than
specified with the {\ufn -mf} option will be truncated.  If the
{\ufn -mf} option was not given, file names will be truncated if they
are too long for the target file system. You may want to shorten long
module names to meaningful abbreviations. This will avoid file name
conflicts for module names that are truncated to the same substring.
Any module name and file name conflicts will be reported.

If your ASN.1 modules have syntactic or semantic errors, each error
will be printed to {\C stderr} along with the file name and line number of
where it occurred. These errors are usable by GNU emacs compiling
tools.  See the next chapter for more information on the types of
errors snacc can detect.

More errors can be detected and reported in a single compile if type
and value definitions are separated by semi-colons.  Separating type
and value definitions with semi-colons is not required, and if used,
need not be used to separate all type and value definitions.
Semi-colons are necessary after some macros that introduce ambiguity.
In general, if you get a parse error you can't figure out, try
separating the surrounding type/value definitions with semicolons.


\subsection{Known Bugs}

\begin{itemize}
  \item
    Snacc has problems with the following case:
    \begin{ASNcode}
      Foo ::= SEQUENCE\\
      \{\+\\
	id IdType,\\
	val ANY DEFINED BY id\-\\
      \}\\
      \\
      IdType ::= CHOICE\\
      \{\+\\
	a INTEGER,\\
	b OBJECT IDENTIFIER\-\\
      \}
    \end{ASNcode}

    The error checking pass will print an error to the effect that the id
    type must be INTEGER or OBJECT IDENTIFER\@.  To fix this you must modify
    the error checking pass as well as the code generation pass.  To be
    cheap about it, disable/fix the error checking and hand modify the
    generated code.

  \item
    The hashing code used for handling ANY DEFINED BY id to type mappings
    will encounter problems if the hash table goes more than four levels
    deep (I think this is unlikely).  To fix this just add linear chaining
    at fourth level.

  \item 
    The {\ufn \dots/configure} script should check whether the machine's floating point format is IEEE or whether the IEEE library exists.

  \item
    The C++ library severly lacks a convenient buffer management class that automatically expands like the C libraries' ExpBuf.
    What use is an efficient buffer management when you have got to build a loop a round snacc's encoding routine that reallocates larger buffers until the result fits?

  \item
    Where this document describes personal experiences, it is usually unclear to which author `I' refers.
    (One way to find out is to look at snacc~1.1's documentation.)

\end{itemize}

\section{\label{bug-section}Reporting Bugs and Your Own Improvements}

Snacc 1.1 was Michael Sample's final release.
While he is watching Snacc's development, he isn't actively developing it himself.

Since there are quite a number of changes from release 1.1 to 1.2rj, bug reports and new features are best sent to me.
I can be reached as \texttt{Robert Joop <rj@rainbow.in-berlin.de>} or \texttt{<rj@fokus.gmd.de>}.
