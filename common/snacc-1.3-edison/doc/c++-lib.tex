% file: .../doc/c++-lib.tex

% $Header: /data/cvs/prototype/story/src/public/tools/snacc-1.3/doc/c++-lib.tex,v 1.1 2006/10/16 09:07:55 joywu Exp $
% $Log: c++-lib.tex,v $
% Revision 1.1  2006/10/16 09:07:55  joywu
% no message
%
% Revision 1.1  1997/01/01 22:47:37  rj
% first check-in
%

\chapter{\label{lib-C++-chapter}C++ ASN.1 Library}
\section{\label{libover-C++-section}Overview}

The following sections describe the C++ representation of the
non-aggregate ASN.1 types, ANY and ANY DEFINED BY types and
the buffer and memory management.  These classes and routines make
up the C++ ASN.1 runtime library. Every aggregate ASN.1 type will be
composed of these library types. The source files for this library
are in {\ufn \dots/c++-lib/inc/} and {\ufn \dots/c++-lib/src/}.

As mentioned in the last chapter, each ASN.1 type is represented by a
C++ class which inherits from the {\C AsnType} base class.  In
addition to the standard encode, decode, print and clone methods
described in the last chapter, each ASN.1 type class in the library
may also have special constructors and other routines that simplify
their use.

Unlike the classes generated for some of the aggregate types such as
SETs and SEQUENCEs, the library types' data members are typically
protected and accessed via methods.

All of the library classes' {\C BDec} routines will report tagging errors
via {\C longjmp()} as described in section~\ref{type-gen-C++-section}.

The top level PDU encode and decode methods are the same for all
library types so they are defined as macros in
{\ufn \dots/c++-lib/inc/asn-config.h}. For clarity's sake, the macro
that is used to define these methods in the library type class
definitions will be replaced with the actual prototypes.

Run the {\ufn test-lib} program in {\ufn \dots/c++-examples/test-lib/}
to make sure the library routines are working properly for your
architecture.  The testing is not exhaustive but should point out
obvious problems.


\section{\label{tag-C++-section}Tags}

The C++ tags are identical to those used in snacc's C ASN.1
environment.  While it would have been nice to define a tag class, the
performance cost would likely have been noticeable.  Also, the snacc
users usually do not have to muck around with tags directly, so the
lack of a class interface will probably not be missed.  The C ASN.1
tags are described in Section~\ref{tag-C-section}.

Initially I defined a C++ class for tags, but close examination of the
produced assembly code led me to reject it.  The C++ class for tags
used the C tag representation internally and had constructor, encode
and decode methods.  The constructor could not be used as
{\C switch} statement case labels like {\C MAKE\_TAG\_ID} because it
did not reduce to an integer constant; this caused problems in the
generated decoders.

As with the C representation of tags, 4 byte long integers limit the
maximum representable tag code to $2^{21}$.  Again, this should not be
a problem.


\section{\label{len-C++-section}Lengths}

The C++ representation of lengths is the same at the C representation
described in Section~\ref{len-C-section}.  The length type was not given
its own C++ class for reasons similar to those of tags.


\section{\label{asntype-C++-section}The AsnType Base Class}

Every ASN.1 type's C++ class uses the {\C AsnType} as its base class.
The {\C AsnType} base class provides the following virtual functions:
\begin{itemize}
  \item the destructor
  \item {\C Clone()}
  \item {\C BDec()}
  \item {\C BEnc()}
  \item {\C Print()}
  \item {\C \_getdesc()} (metacode)
  \item {\C \_getref()} (metacode)
  \item {\C TclGetDesc()} (Tcl interface)
  \item {\C TclGetVal()} (Tcl interface)
  \item {\C TclSetVal()} (Tcl interface)
  \item {\C TclUnsetVal()} (Tcl interface)
\end{itemize}

The {\C AsnType} class is defined as follows:
\begin{Ccode}
class AsnType\\
\{\\
public:\+\\
  virtual			\>\>\~{}AsnType();\\
\\
\<\#ifdef SUPPORT\_ANY\_TYPE\\
  virtual AsnType		\>\>*Clone() const;\\
  virtual void			\>\>BDec (BUF\_TYPE b, AsnLen \&bytesDecoded, ENV\_TYPE env);\\
  virtual AsnLen		\>\>BEnc (BUF\_TYPE b);\\
\<\#else\\
  void				\>\>BDec (BUF\_TYPE b, AsnLen \&bytesDecoded, ENV\_TYPE env) \{\}\\
  AsnLen			\>\>BEnc (BUF\_TYPE b) \{ return 0; \}\\
\<\#endif\\
  virtual void			\>\>Print (ostream \&os) const;\\
\\
\<\#if META\\
  static const AsnTypeDesc	\>\>\_desc;\\
\\
  virtual const AsnTypeDesc	\>\>*\_getdesc() const;\\
  virtual AsnType		\>\>*\_getref (const char *membername, bool create=false);\\
\\
\<private:\\
  const char			\>\>*\_typename() const;\\
\-\\
\#if TCL\\
public:\+\\
  virtual int			\>\>TclGetDesc (Tcl\_DString *) const;\\
  virtual int			\>\>TclGetVal (Tcl\_Interp *) const;\\
  virtual int			\>\>TclSetVal (Tcl\_Interp *, const char *val);\\
  virtual int			\>\>TclUnsetVal (Tcl\_Interp *, const char *membernames);\-\\
\#endif // TCL\\
\#endif // META\\
\};
\end{Ccode}

The {\C AsnType} class and its virtual functions were added to
support the ANY DEFINED BY type handling mechanism.  This mechanism is
described in Section~\ref{any-C++-section}.

Even if you do not use the ANY or ANY DEFINED BY types, the
{\C AsnType} base class may be useful for adding features that are
common to all of the types, such as changing the {\C new} and
{\C delete} functions to improve performance.

Virtual functions provide the simplest method of handling ANY DEFINED
BY and ANY types. Unfortunately, calls to virtual functions are slower
than calls to normal functions due to their indirect nature.  If you
do not need support for the ANY DEFINED BY or ANY types you can remove
most of the virtual functions to improve performance by undefining the
{\C SUPPORT\_ANY\_TYPE} symbol (see the
{\ufn asn-type.h} file).

Note that a virtual destructor is included in the {\C AsnType} base
class as well.  This is done to make sure the {\C delete} routine
always gets the correct size.  See pages 215--217 of Stroustrup
\cite{stroustrup} for a discussion of this.

\section{\label{bool-C++-section}BOOLEAN}

The BOOLEAN type is represented by the {\C AsnBool} class. The
following is the class definition of {\C AsnBool} from the
{\ufn \dots/c++-lib/inc/asn-bool.h} file.

\begin{Ccode}
class AsnBool: public AsnType\\
\{\\
protected:\+\\
  bool				\>\>value;\\
\\
\<public:\\
				\>\>AsnBool (const bool val): value (val) \{\};\\
				\>\>AsnBool() \{\};\\
  AsnType			\>\>*Clone() const;\\
				\>\>operator bool() const \{ return value; \}\\
  AsnBool			\>\>\&operator = (bool newvalue) \{ value = newvalue; return *this; \}\\
\\
  AsnLen			\>\>BEnc (BUF\_TYPE b);\\
  void				\>\>BDec (BUF\_TYPE b, AsnLen \&bytesDecoded, ENV\_TYPE env);\\
\\
  AsnLen			\>\>BEncContent (BUF\_TYPE b);\\
  void				\>\>BDecContent (BUF\_TYPE b, AsnTag tagId, AsnLen elmtLen,\\
					\`AsnLen \&bytesDecoded, ENV\_TYPE env);\\
\\
  int				\>\>BEncPdu (BUF\_TYPE b, AsnLen \&bytesEncoded);\\
  int				\>\>BDecPdu (BUF\_TYPE b, AsnLen \&bytesDecoded);\\
\\
  void				\>\>Print (ostream \&os) const;\\
\\
\<\#if META\\
  static const AsnBoolTypeDesc	\>\>\_desc;\\
\\
  const AsnTypeDesc		\>\>*\_getdesc() const;\\
\\
\<\#if TCL\\
  int				\>\>TclGetVal (Tcl\_Interp *) const;\\
  int				\>\>TclSetVal (Tcl\_Interp *, const char *val);\-\\
\#endif // TCL\\
\#endif // META\\
\};
\end{Ccode}

The upcoming C++ standard \cite{c++-draft} defines a type {\C bool} accompanied by {\C false} and {\C true} to denote the boolean values.
The Snacc configuration script checks whether the C++ compiler already supplies this new type and defines a look-alike if it does not.

For backwards compatibility, {\C AsnBool::true} and {\C AsnBool::false} are still valid.

The {\C operator bool()} is defined such that when an {\C AsnBool}
value is cast to a boolean, it returns the C++ style boolean
value of the {\C AsnBool}s value.  There is also a constructor for {\C AsnBool}
that builds an {\C AsnBool} value from the given C++ style boolean
value. These two methods allow you to manipulate and access
{\C AsnBool} values in a straight forward way as the following code
illustrates.
\begin{Ccode}
Message::Send()\\
\{\+\\
  AsnBool	\>\>okToSend;\\
  bool		\>\>connectionOpen;\\
  bool		\>\>pduOk;\\
  \dots\\
  okToSend = connectionOpen \&\& pduOk; // assign AsnBool from bool\\
  if (okToSend) // cast AsnBool to bool\\
    \>\dots\\
\<\}
\end{Ccode}

The {\C AsnBool} class contains the standard encode and decode
methods that were described in Chapter~\ref{c++-code-gen-chapter}.

{\C BDecContent} will report an error via {\C longjmp} if the
length of an encoded BOOLEAN value's content is not exactly 1 octet.

Note that the {\C Clone} method returns an {\C AsnType~*} value
instead of an {\C AsnBool~*}.  It might be more obvious to return an
{\C AsnBool~*} since due to single inheritance an {\C AsnBool} is also
an {\C AsnType}.  However, it must return an {\C AsnType~*} for it
to override the virtual function {\C Clone} defined in the
{\C AsnType}.

The {\C Print} method will print either ``TRUE'' or ``FALSE''
depending on the {\C AsnBool} value.  No newline or other formatting
characters are printed.  The global indent information does not affect
the output from this method.


\section{\label{int-C++-section}INTEGER}

The INTEGER type is represented by the {\C AsnInt} class. The
following is the class definition of {\C AsnInt} from the
{\ufn \dots/c++-lib/inc/asn-int.h} file.
\begin{Ccode}
class AsnInt: public AsnType\\
\{\\
protected:\+\\
  AsnIntType			\>\>value;\\
\\
\<public:\\
				\>\>AsnInt() \{\}\\
				\>\>AsnInt (AsnIntType val): value (val) \{\}\\
\\
  AsnType			\>\>*Clone() const;\\
\\
				\>\>operator AsnIntType() \{ return value; \}\\
  AsnInt			\>\>\&operator = (AsnIntType newvalue) \{ value = newvalue; return *this; \}
\\
  void				\>\>Set (AsnIntType i) \{ value = i; \}\\
  void				\>\>ReSet (AsnIntType i) \{ value = i; \}\\
\\
  AsnLen			\>\>BEnc (BUF\_TYPE b);\\
  void				\>\>BDec (BUF\_TYPE b, AsnLen \&bytesDecoded, ENV\_TYPE env);\\
\\
  AsnLen			\>\>BEncContent (BUF\_TYPE b);\\
  void				\>\>BDecContent (BUF\_TYPE b, AsnTag tagId, AsnLen elmtLen,\\
					\`AsnLen \&bytesDecoded, ENV\_TYPE env);\\
\\
  int				\>\>BEncPdu (BUF\_TYPE b, AsnLen \&bytesEncoded);\\
  int				\>\>BDecPdu (BUF\_TYPE b, AsnLen \&bytesDecoded);\\
\\
  void				\>\>Print (ostream \&os) const \{ os <\/< value; \}\\
\\
\<\#if META\\
  static const AsnIntTypeDesc	\>\>\_desc;\\
\\
  const AsnTypeDesc		\>\>*\_getdesc() const;\\
\\
\<\#if TCL\\
  int				\>\>TclGetVal (Tcl\_Interp *) const;\\
  int				\>\>TclSetVal (Tcl\_Interp *, const char *val);\-\\
\#endif /* TCL */\\
\#endif /* META */\\
\};
\end{Ccode}

The internal representation of an ASN.1 INTEGER value is a {\C AsnIntType}.
This is a {\C typedef}, the real type may be {\C int}, {\C long} or {\C short}, whatever is found to be 32 bits in size.
The types' sizes depend on the machine and compiler and are determined at configuration time.
The {\C BDecContent} routine will signal an error if the integer
value being decoded will not fit into the {\C AsnIntType}
representation.

Unlike the C ASN.1 library, the non-negative
%{\C unsigned long int}
version of
INTEGER is not provided.  If you need it, it should be relatively
trivial to combine the C unsigned version with the existing C++ class.
The unsigned version of an integer is useful if your ASN.1 source uses
subtyping similar to:
\begin{ASNcode}
Counter ::= [APPLICATION 1] IMPLICIT INTEGER (0..4294967295)
\end{ASNcode}

\section{\label{enum-C++-section}ENUMERATED}

The ENUMERATED type is represented by the {\C AsnEnum} class. The
following is the class definition of {\C AsnEnum} from the
{\ufn \dots/c++-lib/inc/asn-enum.h} file.
\begin{Ccode}
class AsnEnum: public AsnInt\\
\{\\
public:\+\\
\<\#if !TCL\\
				\>\>AsnEnum(): AsnInt() \{\}\\
\<\#endif\\
				\>\>AsnEnum (int i): AsnInt (i) \{\}\\
\\
  AsnType			\>\>*Clone() const;\\
\\
  AsnLen			\>\>BEnc (BUF\_TYPE b);\\
  void				\>\>BDec (BUF\_TYPE b, AsnLen \&bytesDecoded, ENV\_TYPE env);\\
\\
  AsnLen			\>\>BEncContent (BUF\_TYPE b);\\
  void				\>\>BDecContent (BUF\_TYPE b, AsnTag tagId, AsnLen elmtLen,\\
           		                \`AsnLen \&bytesDecoded, ENV\_TYPE env);\\
\\
  int				\>\>BEncPdu (BUF\_TYPE b, AsnLen \&bytesEncoded);\\
  int				\>\>BDecPdu (BUF\_TYPE b, AsnLen \&bytesDecoded);\\
\\
\<\#if META\\
  static const AsnEnumTypeDesc	\>\>\_desc;\\
\\
  const AsnTypeDesc		\>\>*\_getdesc() const;\-\\
\#endif /* META */\\
\};
\end{Ccode}

Note that it is not derived from {\C class AsnType} directly but from {\C class AsnInt} instead.

\section{\label{null-C++-section}NULL}

The NULL type is provided by the {\C AsnNull} class.  This class has
no data members and includes only the standard methods.
\begin{Ccode}
class AsnNull: public AsnType\\
\{\\
public:\+\\
				\>\>AsnNull() \{\}\\
  AsnType			\>\>*Clone() const;\\
\\
  AsnLen			\>\>BEnc (BUF\_TYPE b);\\
  void				\>\>BDec (BUF\_TYPE b, AsnLen \&bytesDecoded, ENV\_TYPE env);\\
\\
  AsnLen			\>\>BEncContent (BUF\_TYPE b);\\
  void				\>\>BDecContent (BUF\_TYPE b, AsnTag tagId, AsnLen elmtLen,\\
					\`AsnLen \&bytesDecoded, ENV\_TYPE env);\\
\\
  int				\>\>BEncPdu (BUF\_TYPE b, AsnLen \&bytesEncoded);\\
  int				\>\>BDecPdu (BUF\_TYPE b, AsnLen \&bytesDecoded);\\
\\
  void				\>\>Print (ostream \&os) const \{ os <\/< "NULL"; \}\\
\\
\<\#if META\\
  static const AsnNullTypeDesc	\>\>\_desc;\\
\\
  const AsnTypeDesc		\>\>*\_getdesc() const;\\
\\
\<\#if TCL\\
  int				\>\>TclGetVal (Tcl\_Interp *) const;\\
  int				\>\>TclSetVal (Tcl\_Interp *, const char *val);\-\\
\#endif /* TCL */\\
\#endif /* META */\\
\};
\end{Ccode}

\section{\label{real-C++-section}REAL}

REAL types are represented by the {\C AsnReal} class.  Internally, a
{\C double} is used to hold the real value.  The following is from
{\ufn \dots/c++-lib/inc/asn-real.h}:
\begin{Ccode}
class AsnReal: public AsnType\\
\{\\
protected:\\
  \>double			\>\>value;\\
\\
public:\+\\
				\>\>AsnReal(): value (0.0) \{\}\\
				\>\>AsnReal (double val): value (val) \{\}\\
  AsnType			\>\>*Clone() const;\\
				\>\>operator double() const  \{ return value; \}\\
  AsnReal			\>\>\&operator = (double newvalue) \{ value = newvalue; return *this; \}\\
\\
  AsnLen			\>\>BEnc (BUF\_TYPE b);\\
  void				\>\>BDec (BUF\_TYPE b, AsnLen \&bytesDecoded, ENV\_TYPE env);\\
\\
  AsnLen			\>\>BEncContent (BUF\_TYPE b);\\
  void				\>\>BDecContent (BUF\_TYPE b, AsnTag tagId, AsnLen elmtLen,\\
					\`AsnLen \&bytesDecoded, ENV\_TYPE env);\\
\\
  int				\>\>BEncPdu (BUF\_TYPE b, AsnLen \&bytesEncoded);\\
  int				\>\>BDecPdu (BUF\_TYPE b, AsnLen \&bytesDecoded);\\
\\
  void				\>\>Print (ostream \&os) const \{ os <\/< value; \}\\
\\
\<\#if META\\
  static const AsnRealTypeDesc	\>\>\_desc;\\
\\
  const AsnTypeDesc		\>\>*\_getdesc() const;\\
\\
\<\#if TCL\\
  int				\>\>TclGetVal (Tcl\_Interp *) const;\\
  int				\>\>TclSetVal (Tcl\_Interp *, const char *val);\-\\
\#endif /* TCL */\\
\#endif /* META */\\
\};\\
\\
extern const AsnReal		\>\>\>PLUS\_INFINITY;\\
extern const AsnReal		\>\>\>MINUS\_INFINITY;
\end{Ccode}

The {\C double} representation and support routines can depend on
the compiler or system you are using so several different encoding
routines are provided.  Even so, you may need to modify the code.

There are three content encoding routines included and they can be selected by defining one of {\C IEEE\_REAL\_FMT}, {\C IEEE\_REAL\_LIB} or nothing.
Defining {\C IEEE\_REAL\_FMT} uses the encode routine that assumes the double representation is the standard IEEE double \cite{68881}.
Defining {\C IEEE\_REAL\_LIB} uses the encode routine that assumes the IEEE functions library (isinf, scalbn, signbit etc.) is available.
If neither are defined, the default encode routine uses {\C frexp}.
%The {\ufn \dots/configure} script tries to find the {\C isinf} and {\C finite} functions 
%Look at the related comments in {\ufn \dots/c++-lib/inc/asn-config.h}.
Currently, the {\ufn \dots/configure} script has not got any checks for the IEEE format or library and therefore does not define any of the symbols. (This should be fixed.)

{\C AsnReal} constants are used to hold {\C PLUS\_INFINITY} and
{\C MINUS\_INFINITY} values.  These values are initialized using the
{\C AsnReal} constructor mechanism with the {\C AsnPlusInfinity}
and {\C AsnMinusInfinity} routines.  If you do not define
{\C IEEE\_REAL\_FMT} or {\C IEEE\_REAL\_LIB}, you should rewrite the
{\C AsnPlusInfinity} routine such that it is correct for your system.

There is only one content decoding routine and it builds the value
through multiplication and the {\C pow} routine (requires the math
library). The content decoding routine only supports the binary
encoding of a REAL, not the decimal encoding.

\section{\label{bits-C++-section}BIT STRING}

The BIT STRING type is represented by the {\C AsnBits} class.  From\linebreak
{\ufn \dots/c++-lib/inc/asn-bits.h}:
\begin{Ccode}
class AsnBits: public AsnType\\
\{\\
private:\+\\
  int				\>\>BitsEquiv (AsnBits \&ab);\\
  void				\>\>BDecConsBits (BUF\_TYPE b, AsnLen elmtLen,\\
					\`AsnLen \&bytesDecoded, ENV\_TYPE env);\\
  void				\>\>FillBitStringStk (BUF\_TYPE b, AsnLen elmtLen0,\\
					\`AsnLen \&bytesDecoded, ENV\_TYPE env);\\
\<protected:\\
  size\_t			\>\>bitLen;\\
  char				\>\>*bits;\\
\\
\<public:\\
				\>\>AsnBits() \{ bits = NULL; bitLen = 0; \}\\
				\>\>AsnBits (const size\_t numBits) \{ Set (numBits); \}\\
				\>\>AsnBits (const char *bitOcts, const size\_t numBits) \\
					\`\{ Set (bitOcts, numBits); \}\\
				\>\>AsnBits (const AsnBits \&b) \{ Set (b); \}\\
				\>\>\~{}AsnBits() \{ delete bits; \}\\
\\
  AsnType			\>\>*Clone() const;\\
\\
  AsnBits			\>\>\&operator = (const AsnBits \&b) \{ ReSet (b); return *this; \}\\
\\
  size\_t			\>\>BitLen() \{ return bitLen; \}\\
\\
  bool				\>\>operator == (AsnBits \&ab) const \{ return BitsEquiv (ab); \}\\
  bool				\>\>operator != (AsnBits \&ab) const \{ return !BitsEquiv (ab); \}\\
\\
  // overwrite existing bits and bitLen values\\
  void				\>\>Set (size\_t numBits);\\
  void				\>\>Set (const char *bitOcts, size\_t numBits);\\
  void				\>\>Set (const AsnBits \&b);\\
\\
  // free old bits value, the reset bits and bitLen values\\
  void				\>\>ReSet (const size\_t numBits);\\
  void				\>\>ReSet (const char *bitOcts, size\_t numBits);\\
  void				\>\>ReSet (const AsnBits \&b);\\
\\
  void				\>\>SetBit (size\_t);\\
  void				\>\>ClrBit (size\_t);\\
  int				\>\>GetBit (size\_t) const;\\
\\
  AsnLen			\>\>BEnc (BUF\_TYPE b);\\
  void				\>\>BDec (BUF\_TYPE b, AsnLen \&bytesDecoded, ENV\_TYPE env);\\
\\
  AsnLen			\>\>BEncContent (BUF\_TYPE b);\\
  void				\>\>BDecContent (BUF\_TYPE b, AsnTag tagId, AsnLen elmtLen,\\
					\`AsnLen \&bytesDecoded, ENV\_TYPE env);\\
\\
  int				\>\>BEncPdu (BUF\_TYPE b, AsnLen \&bytesEncoded);\\
  int				\>\>BDecPdu (BUF\_TYPE b, AsnLen \&bytesDecoded);\\
\\
  void				\>\>Print (ostream \&os) const;\\
\\
\<\#if META\\
  static const AsnBitsTypeDesc	\>\>\_desc;\\
\\
  const AsnTypeDesc		\>\>*\_getdesc() const;\\
\\
\<\#if TCL\\
  int				\>\>TclGetVal (Tcl\_Interp *) const;\\
  int				\>\>TclSetVal (Tcl\_Interp *, const char *val);\-\\
\#endif /* TCL */\\
\#endif /* META */\\
\};
\end{Ccode}

The {\C AsnBits} class contains a pointer to the bits and an
integer that holds the length in bits of the BIT STRING\@.

In addition to the standard methods, the {\C AsnBits} class has
methods for initializing and comparing bit string values and methods
for setting and getting individual bits in a value.

An {\C AsnBits} value can be created three ways: from the number of
bits, from a {\C char~*} and its bit length or from another
{\C AsnBits} value.  Look at the constructors and the {\C Set} and
{\C ReSet} methods.

{\C SetBit} and {\C ClrBit} can be used for setting the values of
individual bits in the BIT STRING value.  Given the bit's index,
{\C SetBits} sets that bit to one.  {\C ClrBit} sets the bit
of the given index to zero.  The bit indexes start at zero, with zero
being the first (most signficant) bit in the BIT STRING\@. {\C GetBit}
will return {\C true} if the specified bit is one and {\C false} if the bit is
zero.  If the given bit index is too large, {\C SetBit} and
{\C ClrBit} do nothing and {\C GetBit} returns {\C false}\@.

The {\C ==} and {\C !=} operators have been overloaded such that
given two {\C AsnBits} values, they will behave as expected.

Each {\C AsnBits} value stores its bit string in a single contiguous
block of memory.  Received BIT STRING values that were encoded in the
constructed form are converted to the simple, flat form (see Section
\ref{bits-C-section}).  Snacc provides no facility for encoding or
internally representing constructed BIT STRING values.

\section{\label{octets-C++-section}OCTET STRING}
OCTET STRING values are represented with the {\C AsnOcts} class.
From\linebreak {\ufn \dots/c++-lib/inc/asn-octs.h}:
\begin{Ccode}
class AsnOcts: public AsnType\\
\{\\
private:\+\\
  int				\>\>OctsEquiv (const AsnOcts \&o);\\
  void				\>\>FillBitStringStk (BUF\_TYPE b, AsnLen elmtLen0,\\
					\`AsnLen \&bytesDecoded, ENV\_TYPE env);\\
  void				\>\>BDecConsOcts (BUF\_TYPE b, AsnLen elmtLen,\\
					\`AsnLen \&bytesDecoded, ENV\_TYPE env);\\
\\
\<protected:\\
  size\_t			\>\>octetLen;\\
  char				\>\>*octs;\\
\\
\<public:\\
  // constructor and Set alway copy strings so destructor can always delete\\
				\>\>AsnOcts(): octs (NULL), octetLen (0) \{\}\\
				\>\>AsnOcts (const char *str)  \{ Set (str); \}\\
				\>\>AsnOcts (const char *str, const size\_t len) \{ Set (str, len); \}\\
				\>\>AsnOcts (const AsnOcts \&o) \{ Set (o); \}\\
				\>\>\~{}AsnOcts() \{ delete octs; \}\\
  AsnType			\>\>*Clone() const;\\
\\
  AsnOcts			\>\>\&operator = (const AsnOcts \&o) \{ ReSet (o); return *this; \}\\
  AsnOcts			\>\>\&operator = (const char *str) \{ ReSet (str); return *this; \}\\
\\
  size\_t			\>\>Len() const \{ return octetLen; \}\\
				\>\>operator const char *() const \{ return octs; \}\\
				\>\>operator char *() \{ return octs; \}\\
\\
  bool				\>\>operator == (const AsnOcts \&o) const \{ return OctsEquiv (o); \}\\
  bool				\>\>operator != (const AsnOcts \&o) const \{ return !OctsEquiv (o); \}\\
\\
  // these set the octs and octetLen values\\
  void				\>\>Set (const char *str, size\_t len);\\
  void				\>\>Set (const AsnOcts \&o);\\
  void				\>\>Set (const char *str);\\
\\
  // these free the old octs value and then reset the octs and octetLen values\\
  void				\>\>ReSet (const char *str, size\_t len);\\
  void				\>\>ReSet (const AsnOcts \&o);\\
  void				\>\>ReSet (const char *str);\\
\\
  AsnLen			\>\>BEnc (BUF\_TYPE b);\\
  void				\>\>BDec (BUF\_TYPE b, AsnLen \&bytesDecoded, ENV\_TYPE env);\\
\\
  AsnLen			\>\>BEncContent (BUF\_TYPE b);\\
  void				\>\>BDecContent (BUF\_TYPE b, AsnTag tagId, AsnLen elmtLen,\\
					\`AsnLen \&bytesDecoded, ENV\_TYPE env);\\
\\
  int				\>\>BEncPdu (BUF\_TYPE b, AsnLen \&bytesEncoded);\\
  int				\>\>BDecPdu (BUF\_TYPE b, AsnLen \&bytesDecoded);\\
\\
  void				\>\>Print (ostream \&os) const;\\
\\
\<\#if META\\
  static const AsnOctsTypeDesc	\>\>\_desc;\\
\\
  const AsnTypeDesc		\>\>*\_getdesc() const;\\
\\
\<\#if TCL\\
  int				\>\>TclGetVal (Tcl\_Interp *) const;\\
  int				\>\>TclSetVal (Tcl\_Interp *, const char *val);\-\\
\#endif /* TCL */\\
\#endif /* META */\\
\};
\end{Ccode}

The {\C AsnOcts} class contains a pointer to the octets and an
integer that holds the length in octets of the OCTET STRING\@.

There are four constructors for {\C AsnOcts}.  The parameterless
constructor will initialize the octet string to zero length with a
{\C NULL} octets pointer.  The constructor that takes a single {\C char~*}
assumes that the given string is NUL terminated and initializes the octet
pointer with a pointer to a copy of the given string and sets the
{\C octetLen} to the {\C strlen} of the string (this does not usually
include the NUL terminator).  The constructor that takes
{\C char~*} and a length, {\C len}, initializes the octets pointer
to point to a copy of {\C len} characters from the given string and
sets the {\C octetLen} to {\C len}.  The last constructor will initialize
an {\C AsnOcts} value by copying the given {\C AsnOcts} value.

As with the BIT STRING content decoder, OCTET STRING content decoder
can handle constructed values.  These are handled in the same way as
the constructed BIT STRING values; they are converted to the simple
contiguous representation.  Every OCTET STRING value will
automatically have a NUL terminator appended to it; this extra
character will not be included in the string's length and will make
some strings easier to deal with for printing etc.

The {\C operator char~*()} is defined for the {\C AsnOcts} class to
return a pointer to the octets.  The {\C Len}
method returns the length in bytes of the string value.
These may be useful for passing the
octets to other functions such as {\C memcpy} etc.  

The {\C ==} and {\C !=} operators have been overloaded such that
given two {\C AsnOcts} values, they will behave as expected.



\section{\label{oid-C++-section}OBJECT IDENTIFIER}
OBJECT IDENTIFIER values are represented with the {\C AsnOid} class.
From\linebreak {\ufn \dots/c++-lib/inc/asn-oid.h}:
\begin{Ccode}
class AsnOid: public AsnType\\
\{\\
private:\\
  \>int				\>\>OidEquiv (AsnOid o);\\
\\
protected:\+\\
  size\_t			\>\>octetLen;\\
  char				\>\>*oid;\\
\\
\<public:\\
				\>\>AsnOid(): oid (NULL), octetLen (0) \{\}\\
				\>\>AsnOid (const char *encOid, size\_t len) \{ Set (encOid, len); \}\\
				\>\>AsnOid (const AsnOid \&o) \{ Set (o); \}\\
				\>\>AsnOid (unsigned long int a1, unsigned long int a2, long int a3 = -1,\\
					\`long int a4 = -1, long int a5 = -1, long int a6 = -1, long int a7 = -1,\\
					\`long int a8 = -1, long int a9 = -1, long int a10 = -1, long int a11 = -1);\\
				\>\>\~{}AsnOid() \{ delete oid; \}\\
  AsnType			\>\>*Clone() const;\\
\\
  AsnOid			\>\>\&operator = (const AsnOid \&o) \{ ReSet (o); return *this; \}\\
\\
  size\_t			\>\>Len() \{ return octetLen; \}\\
  const char			\>\>*Str() const \{ return oid; \}\\
				\>\>operator const char * () const \{ return oid; \}\\
				\>\>operator char * () \{ return oid; \}\\
  unsigned long int		\>\>NumArcs() const;\\
\\
  bool				\>\>operator == (AsnOid \&o) const \{ return OidEquiv (o); \}\\
  bool				\>\>operator != (AsnOid \&o) const \{ return !OidEquiv (o); \}\\
\\
  // Set methods overwrite oid and octetLen values\\
  void				\>\>Set (const char *encOid, const size\_t len);\\
  void				\>\>Set (const AsnOid \&o);\\
\\
  // first two arc numbers are mandatory.  rest are optional since negative arc nums are not allowed in the\\
  // encodings, use them to indicate the 'end of arc numbers' in the optional parameters\\
  void				\>\>Set (unsigned long int a1, unsigned long int a2, long int a3 = -1,\\
					\`long int a4 = -1, long int a5 = -1, long int a6 = -1, long int a7 = -1,\\
					\`long int a8 = -1, long int a9 = -1, long int a10 = -1, long int a11 = -1);\\
\\
  // ReSet routines are like Set except the old oid value is freed\\
  void				\>\>ReSet (const char *encOid, const size\_t len);\\
  void				\>\>ReSet (const AsnOid \&o);\\
  void				\>\>ReSet (unsigned long int a1, unsigned long int a2, long int a3 = -1,\\
					\`long int a4 = -1, long int a5 = -1, long int a6 = -1, long int a7 = -1,\\
					\`long int a8 = -1, long int a9 = -1, long int a10 = -1, long int a11 = -1);\\
\\
  AsnLen			\>\>BEnc (BUF\_TYPE b);\\
  void				\>\>BDec (BUF\_TYPE b, AsnLen \&bytesDecoded, ENV\_TYPE env);\\
\\
  AsnLen			\>\>BEncContent (BUF\_TYPE b);\\
  void				\>\>BDecContent (BUF\_TYPE b, AsnTag tagId, AsnLen elmtLen,\\
					\`AsnLen \&bytesDecoded, ENV\_TYPE env);\\
\\
  int				\>\>BEncPdu (BUF\_TYPE b, AsnLen \&bytesEncoded);\\
  int				\>\>BDecPdu (BUF\_TYPE b, AsnLen \&bytesDecoded);\\
\\
  void				\>\>Print (ostream \&os) const;\\
\\
\<\#if META\\
  static const AsnOidTypeDesc	\>\>\_desc;\\
\\
  const AsnTypeDesc		\>\>*\_getdesc() const;\\
\\
\<\#if TCL\\
  int				\>\>TclGetVal (Tcl\_Interp *) const;\\
  int				\>\>TclSetVal (Tcl\_Interp *, const char *val);\-\\
\#endif /* TCL */\\
\#endif /* META */\\
\};
\end{Ccode}

The {\C AsnOid} stores OBJECT IDENTIFIER values in their encoded form
to improve performance.  It seems that the most common operation with
OBJECT IDENTIFIERs is to compare for equality, for which the encoded
representation (which is canonical) works well.

The {\C AsnOid} is very similar to the {\C AsnOcts} class in all
respects, except that its content decoding routine does not need to
handle constructed encodings.

The {\C AsnOid} class has four constructors which are similiar to
those of the {\C AsnOcts} class.  A special constructor that takes
arc numbers as parameters and uses default parameters is provided.  An
OBJECT IDENTIFIER value must have at least two arc numbers so the
first two parameters do not have default values.  All of the other
parameters are optional; since their default value of {\C --1} is an
invalid arc number (they must be positive) they will not be used in the
value. For example to build the value {\C \{1 2 3\}} you simply use
{\C AsnOid (1, 2, 3)}.  This constructor is convenient but is more
expensive in terms of CPU time than the others.

The {\C operator char~*()} is defined for the {\C AsnOid} class to
return a pointer to the encoded OBJECT IDENTIFIER value.  The {\C Len}
method returns the length in bytes of the encode OBJECT IDENTIFIER
value (NOT the number arcs in the value).  These may be useful for
passing the octets to other functions such as {\C memcpy} etc.
{\C NumArcs} returns the number of arcs that the value is comprised of.

The {\C ==} and {\C !=} operators have been overloaded such that
given two {\C AsnOcts} values, they will behave as expected.



\section{\label{list-C++-section}SET OF and SEQUENCE OF}

In the C ASN.1 library, the list type was in the library because it
was generic and every SET OF and SEQUENCE OF was defined as an
{\C AsnList}.  In C++, a new class is defined every list, providing
a type safe list mechanism.  This was described in the previous chapter.


\section{\label{any-C++-section}ANY and ANY DEFINED BY}

The ANY DEFINED BY type can be handled automatically by snacc
provided you use the SNMP OBJECT-TYPE macro to specify the identifier
to type mappings.  The identifier can be an INTEGER or OBJECT
IDENTIFIER\@.  Handling ANY types properly will require modifications to
the generated code since there is no identifier associated with the
type.

Look at the C and C++ ANY examples and the {\ufn any.asn1} file
included with this release for information on using the OBJECT-TYPE
macro.  Note that the OBJECT-TYPE macro has been modified slightly to
allow INTEGER values (identifiers).

An ANY DEFINED BY type is represented by the {\C AsnAny} class.
The following is from {\ufn \dots/c++-lib/inc/asn-any.h}.
\begin{Ccode}
/* AnyInfo is a hash table entry */\\
class AnyInfo\\
\{\\
public:\+\\
  int				\>\>anyId;  // will be a value from the AnyId enum\\
  AsnOid			\>\>oid;    // will be zero len/null if intId is valid\\
  AsnInt			\>\>intId;\\
  AsnType			\>\>*typeToClone;\-\\
\};\\
\\
class AsnAny: public AsnType\\
\{\\
public:\+\\
  static Table			\>\>*oidHashTbl;  // all AsnAny class instances\\
  static Table			\>\>*intHashTbl;  // share these tables\\
  AnyInfo			\>\>*ai; // points to entry in hash tbl for this type\\
  AsnType			\>\>*value;\\
\\
				\>\>AsnAny() \{ ai = NULL; value = NULL; \}\\
\\
  // class level methods\\
  static void			\>\>InstallAnyByInt (AsnInt intId, int anyId, AsnType *type);\\
  static void			\>\>InstallAnyByOid (AsnOid \&oid, int anyId, AsnType *type);\\
\\
  int				\>\>GetId() \{ return ai ? ai-->anyId : -1; \}\\
  void				\>\>SetTypeByInt (AsnInt id);\\
  void				\>\>SetTypeByOid (AsnOid \&id);\\
\\
  AsnLen			\>\>BEnc (BUF\_TYPE b);\\
  void				\>\>BDec (BUF\_TYPE b, AsnLen \&bytesDecoded, ENV\_TYPE env);\\
\\
  int				\>\>BEncPdu (BUF\_TYPE b, AsnLen \&bytesEncoded);\\
  int				\>\>BDecPdu (BUF\_TYPE b, AsnLen \&bytesDecoded);\\
\\
  void				\>\>Print (ostream \&os) const \{ value-->Print (os); \}\\
\<\};
\end{Ccode}

The C++ mechanism is similar to the C mechanism which uses hash tables
to hold the identifier to type mappings.  In this section we will
discuss the main differences of the C++ ANY DEFINED BY handling
mechanism.  You should read Section~\ref{any-C-section} for caveats and
other important information.

In C, the hash table entry held the size of the type and pointers to
its encode, decode, free etc. routines to describe the type.  In C++
these have been replaced with a pointer to an instance of the type. A
hash table entry contains:
\begin{itemize}
\item {the {\C anyId}}
\item {the INTEGER or OBJECT IDENTIFIER that maps to it}
\item {a pointer to an instance of the identified type}
\end{itemize}

All C++ ASN.1 types use the {\C AsnType} base class which designates the following functions as virtual:
\begin{itemize}
  \item the destructor
  \item {\C Clone()}
  \item {\C BDec()}
  \item {\C BEnc()}
  \item {\C Print()}
  \item {\C \_getdesc()} (metacode)
  \item {\C \_getref()} (metacode)
  \item {\C TclGetDesc()} (Tcl interface)
  \item {\C TclGetVal()} (Tcl interface)
  \item {\C TclSetVal()} (Tcl interface)
  \item {\C TclUnsetVal()} (Tcl interface)
\end{itemize}

This allows the ANY DEFINED BY handling routines to treat a value of
any ASN.1 type as an {\C AsnType}.  So, for each type the ANY
DEFINED BY handling code has access to the virtual methods.  Note
that the {\C value} field in the {\C AsnAny} class and the
{\C typeToClone} field in the {\C AnyInfo} class are both
{\C AsnType~*}.

To build an ANY DEFINED BY value, simply set the value field in the
{\C AsnAny} object to point to the object you want to encode.  Then
set the identifier field for that ANY DEFINED BY value to the correct
identifier (as generated for its OBJECT-TYPE macro value).  It is very
important to do this correctly because the encoder will simply call
the virtual {\C BEnc} routine for the object pointed to by the
{\C AsnAny}'s value field.  There is no attempt to make sure that
the identifier field's value matches the object that was encoded.

A potential solution to the last identifier problem is to add a type
id field to the {\C AsnType} base class.  Snacc could generate a
unique identifier (that would be stored in the base class) for each
type.  The encoder could then check identifiers between the value
being encoded and the value stored in the hash table.  The identifier
in the base class could easily be automatically set (correctly) from
the constructors for each type (constructors are snacc generated).  It
would be difficult to ensure unique identifiers for each type between
modules if the ASN.1 modules were compiled separately.

Before an ANY DEFINED BY value can be decoded, the field that contains
its identifier must have been decoded and used with the {\C AsnAny}
value's {\C SetTypeByInt} or {\C SetTypeByOid} methods.  Then the
ANY DEFINED BY value can be decoded by calling its ({\C AsnAny})
{\C BDec} routine.  This in turn calls the {\C Clone} routine on
the type in the hash table entry to generate the correct object.  Then
the {\C BDec} method of the newly created object is called.

When the C ANY DEFINED BY decoder allocates a value, it uses the size
information for the identified type.  This is not safe for C++ so the
virtual {\C Clone} routine was added to the {\C AsnType} base
class.  This allows the proper constructor mechanism to be used when
allocating the value.

The virtual {\C Clone} routine simply calls its type's parameterless
constructor via {\C new} (hence every ASN.1 type's class must have a
parameterless constructor).  {\C Clone} is a poor name since
the routine only produces a new instance of the given type without
copying the original's data.

The hash tables are automatically initialized using the C++ constructor
mechanism.  You do not need to call any initialization routines as
described in the C chapter.


\section{\label{buffer-C++-section}Buffer Management}

The C++ buffer management provided with snacc is similar to that of
the C {\C SBuf}s.  The following is from
{\ufn \dots/c++-lib/inc/asn-buf.h}:
\begin{Ccode}
class AsnBuf\\
\{\\
protected:\+\\
  char			\>\>*dataStart;\\
  char			\>\>*dataEnd;\\
  char			\>\>*blkStart;\\
  char			\>\>*blkEnd;\\
  char			\>\>*readLoc;\\
  bool			\>\>writeError;\\
  bool			\>\>readError;\\
\\
\<public:\\
  void			\>\>Init (char *data, size\_t dataLen);\\
  void			\>\>ResetInReadMode();\\
  void			\>\>ResetInWriteRvsMode();\\
  void			\>\>InstallData (char *data, size\_t dataLen);\\
  size\_t		\>\>DataLen();\\
  char			\>\>*DataPtr();\\
  size\_t		\>\>BlkLen();\\
  char			\>\>*BlkPtr();\\
  bool			\>\>Eod();\\
\\
  /* the following 9 methods are required */\\
  bool			\>\>ReadError();\\
  bool			\>\>WriteError();\\
  void			\>\>Skip (size\_t skipLen);\\
  size\_t		\>\>CopyOut (char *dst, size\_t copyLen);\\
  unsigned char		\>\>PeekByte();\\
  char			\>\>*GetSeg (size\_t *lenPtr);\\
  void			\>\>PutSegRvs (char *seg, size\_t segLen);\\
  unsigned char		\>\>GetByte();\\
  void			\>\>PutByteRvs (unsigned char byte);\\
\<\};
\end{Ccode}

This is the only buffer type provided with the C++ library.

\section{\label{mem-C++-section}Dynamic Memory Management}

The built-in C++ memory management system is used by snacc ({\C new}
and {\C delete}).  Better performance might be gained by using a
different management scheme.

To change {\C new} and {\C delete} to use your own memory
management scheme the best way to start is by defining them as virtual
in the {\C AsnType} base class.  More information on providing your
own memory management can be found in Stroustrup \cite{stroustrup}.

\section{\label{error-C++-section}Error Management}

The C++ ASN.1 error management is identical to that of the C ASN.1
model.  C++ exception handling ({\C try} and {\C throw}) were not
used to replace {\C setjmp()} and {\C longjmp()} because they were not
implemented by the C++ compiler we used.
